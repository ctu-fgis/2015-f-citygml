\documentclass{beamer}

\mode<presentation> {
  \usetheme{Goettingen}
  \setbeamercovered{transparent}
}

\usepackage[utf8]{inputenc}
\usepackage[czech]{babel}
%\usepackage{graphicx}
\definecolor{thisblue}{RGB}{51 51 179}
\hypersetup{allcolors=thisblue,colorlinks=true}

\title{cityglm2stl}
\author{Miro Hrončok}
\institute[IF2015]{FSv ČVUT}


\begin{document}

\begin{frame}
  \titlepage
\end{frame}

\section{Zadání}

\begin{frame}
  \frametitle{Zadání}
    \begin{itemize}[<+->]
      \item foo
    \end{itemize}
\end{frame}

\section{CityGML}

\begin{frame}
  \frametitle{Co to je CityGML}
    \begin{itemize}[<+->]
      \item foo
    \end{itemize}
\end{frame}

\section{Vnitřnosti CityGML}

\begin{frame}
  \frametitle{Z čeho se skládá CityGML}
    \begin{itemize}[<+->]
      \item foo
    \end{itemize}
\end{frame}

\section{3D objekty v CityGML}

\begin{frame}
  \frametitle{Reprezentace 3D objektů v CityGML}
    \begin{itemize}[<+->]
      \item foo
    \end{itemize}
\end{frame}

\section{STL}

\begin{frame}
  \frametitle{Co je STL}
    \begin{itemize}[<+->]
      \item foo
    \end{itemize}
\end{frame}

\section{Vnitřnosti STL}

\begin{frame}
  \frametitle{Z čeho se skládá STL}
    \begin{itemize}[<+->]
      \item foo
    \end{itemize}
\end{frame}

\section{Jak to funguje}

\begin{frame}
  \frametitle{Jak funguje cityglm2stl}
    \begin{itemize}[<+->]
      \item foo
    \end{itemize}
\end{frame}

\section{Výsledky a problémy}

\begin{frame}
  \frametitle{Co se podařilo, co ne}
    \begin{itemize}[<+->]
      \item foo
    \end{itemize}
\end{frame}

\end{document}

