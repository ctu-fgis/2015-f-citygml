\documentclass{beamer}

\mode<presentation> {
  \usetheme{Goettingen}
  \setbeamercovered{transparent}
}

\usepackage[utf8]{inputenc}
\usepackage[czech]{babel}
%\usepackage{graphicx}
\definecolor{thisblue}{RGB}{51 51 179}
\hypersetup{allcolors=thisblue,colorlinks=true}

\title{cityglm2stl}
\author{Miro Hrončok}
\institute[IF2015]{FSv ČVUT}


\begin{document}

\begin{frame}
  \titlepage
\end{frame}

\section{Zadání}

\begin{frame}
  \frametitle{Zadání}
    \begin{itemize}[<+->]
      \item knihovna pro Python
      \item převod 3D objektů z CityGML souborů do formátu STL
      \item nějaké lidské rozhraní
    \end{itemize}
\end{frame}

\section{CityGML}

\begin{frame}
  \frametitle{Co to je CityGML}
    \begin{itemize}[<+->]
      \item \href{http://www.citygml.org/}{citygml.org}
      \item souborový formát na bázi XML
      \item otevřený standard
      \item virtuální 3D modely měst a/nebo krajiny
      \item mnoho metainformací
      \item hlavně pro zobrazení na obrazovce
      \item různé verze standardu
    \end{itemize}
\end{frame}

\section{Vnitřnosti CityGML}

\begin{frame}
  \frametitle{Z čeho se skládá CityGML}
    \begin{itemize}[<+->]
      \item obří XML strom
      \item obsahuje „city objekty“
      \item každý 3D city objekt má mj. definovaný svůj tvar
      \item mnoho dalších informací, pro tuto práci nepotřebných
      \begin{itemize}[<+->]
        \item textury
        \item materiály
        \item meta informace
      \end{itemize}
    \end{itemize}
\end{frame}

\section{3D objekty v CityGML}

\begin{frame}
  \frametitle{Reprezentace 3D objektů v CityGML}
    \begin{itemize}[<+->]
      \item skládají se z polygonů
      \begin{itemize}[<+->]
        \item jedno ohraničení exteriéru polygonu
        \item libovolně mnoho ohraničení interiérů polygonu
      \end{itemize}
      \item seznam bodů v 3D prostoru
      \item orientace?
    \end{itemize}
\end{frame}

\section{STL}

\begin{frame}
  \frametitle{Co je STL}
    \begin{itemize}[<+->]
      \item reprezentace 3D objektů
      \item hlavně pro 3D tisk
      \item žádné metainformace (kromě názvu)
      \item binární/textová podoba
      \item de facto standard
    \end{itemize}
\end{frame}

\section{Vnitřnosti STL}

\begin{frame}
  \frametitle{Z čeho se skládá STL}
    \begin{itemize}[<+->]
      \item triangulární síť (mesh)
      \item seznam trojúhelníků
      \begin{itemize}[<+->]
        \item seznam vrcholů
        \item normála
      \end{itemize}
    \end{itemize}
\end{frame}

\section{Jak to funguje}

\begin{frame}
  \frametitle{Jak funguje cityglm2stl}
    \begin{enumerate}[<+->]
      \item rozparsování XML stromu
      \item nalezení city objektů
      \item extrakce polygonů
      \item triangulace polygonů (poly2tri)
      \item export textového STL souboru
    \end{enumerate}
\end{frame}

\section{Výsledky a problémy}

\begin{frame}
  \frametitle{Co se podařilo, co ne}
    \begin{itemize}[<+->]
      \item \href{https://github.com/ctu-yfsg/2015-f-citygml}{citygml2stl na GitHubu}
      \item \href{http://pypi.python.org/pypi/citygml2stl}{citygml2stl na PyPI}
      \item knihovní rozhraní s mnoho možnostmi
      \item jednoduché rozhraní pro příkazovou řádku
      \item licence MIT
      \item velké pokrytí testy
      \item poly2tri má problém triangulovat některé polygony
      \item výsledky nejsou validní STL meshe
    \end{itemize}
\end{frame}

\end{document}

